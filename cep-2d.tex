\documentclass{patmorin}
\listfiles
\usepackage{pat}
\usepackage[T1]{fontenc}
\usepackage[utf8]{inputenc}
\usepackage{paralist}
\usepackage[normalem]{ulem}
\usepackage{mathtools}

\usepackage{todonotes}
\usepackage{comment}

% david proposes the following additions
% \renewcommand{\ge}{\geqslant}
% \renewcommand{\le}{\leqslant}
% \renewcommand{\geq}{\geqslant}
% \renewcommand{\leq}{\leqslant}

\newcommand{\vida}[1]{{\color{DarkGreen} Vida: #1}}
\newcommand{\pat}[1]{\textcolor{Blue}{Pat: #1}}
\newcommand{\gwen}[1]{\textcolor{Purple}{Gwen: #1}}
\newcommand{\piotr}[1]{\textcolor{red}{Piotr: #1}}

% \numberwithin{equation}{lem}

\crefname{p}{}{}
\creflabelformat{p}{#2(#1)#3}

\newenvironment{clmproof}{\begin{proof}[Proof of Claim:]\renewcommand{\qedsymbol}{\rule{1ex}{1ex}}}{\end{proof}}

\usepackage[longnamesfirst,numbers,sort&compress]{natbib}

% \newcommand{\mathdefin}[1]{\color{brightmaroon}#1}}
\setlength{\parskip}{1ex}

% Document-specific commands and math operators
\DeclareMathOperator{\tw}{tw}
\DeclareMathOperator{\pw}{pw}
\DeclareMathOperator{\bw}{bw}
\DeclareMathOperator{\td}{td}
\DeclareMathOperator{\diam}{diam}
\DeclareMathOperator{\radius}{radius}
\DeclareMathOperator{\pth}{path}
\DeclareMathOperator{\mindist}{min-dist}
\DeclareMathOperator{\mindeg}{min-deg}
\DeclareMathOperator{\girth}{girth}
\DeclareMathOperator{\dist}{dist}
\DeclareMathOperator{\ld}{ld}
\DeclareMathOperator{\polylog}{polylog}
\DeclareMathOperator{\evol}{Evol}
\DeclareMathOperator{\ivol}{Ivol}
\DeclareMathOperator{\tvol}{Tvol}
\newcommand{\NN}{\mathbb{N}}
\newcommand{\GG}{\mathcal{G}}
\newcommand{\Oh}{\mathcal{O}}
\DeclareMathOperator{\thick}{th}

\DeclarePairedDelimiter\set{\{}{\}}

\title{\MakeUppercase{{E}rdős–{P}ósa property of cycles that are far apart}}

%\title{\MakeUppercase{\boldmath Planar graphs are contained in $\tilde{O}(\sqrt{n})$-blowups of fans}}

%Fan-Partitions of Planar Graphs (and Beyond)  \newline by Local Sparsification and Volume-Preserving Embeddings}}

\author{
 Vida Dujmovi{\'c}\,\footnote{School of Computer Science and Electrical Engineering, University of Ottawa, Ottawa, Canada (\texttt{vida.dujmovic@uottawa.ca}). Research supported by NSERC and a University of Ottawa Research Chair.}
 \qquad
 Gwena\"el Joret\footnote{D\'epartement d'Informatique, Universit\'e libre de Bruxelles, Belgium ({\tt gwenael.joret@ulb.be}). G.\ Joret is supported by the Belgian National Fund for Scientific Research (FNRS) and by the Australian Research Council.}
 \qquad
 Piotr Micek\footnote{Department of Theoretical Computer Science, Jagiellonian University, Kraków, Poland (\texttt{piotr.micek@uj.edu.pl}). Research supported
 the National Science Center of Poland under grant UMO-2018/31/G/ST1/03718 within the BEETHOVEN program.}
 \qquad
 Pat Morin\footnote{School of Computer Science, Carleton University, Ottawa, Canada (\texttt{morin@scs.carleton.ca}). Research supported by NSERC and the Ontario Ministry of Research and Innovation.}}

\date{}


\begin{document}

\maketitle

\begin{abstract}
  We prove that there exists a function $f:\mathbb{N}\to\mathbb{N}$ such that for all nonnegative integers $k$ and $d$,  for every graph $G$,  either $G$ contains $k$ cycles such that vertices of different cycles have distance greater than $d$ in $G$, or there exists a subset $X$ of vertices of $G$ with $|X|\leq f(k)$ such that  $G-B_G(X,100d)$ is a forest.
\end{abstract}

\section{Introduction}

Let $G$ be a graph and let $d$ be a non-negative integer.
A set $\mathcal{C}$ of cycles in $G$ is a \defin{$d$-packing of cycles in $G$} if $\dist_G(V(C),V(C'))> d$ for every two distinct $C,C'\in\mathcal{C}$.
%The \defin{distance-$d$ $\rho$acking number} of cycles in $G$ is the maximum integer $p$ such that $G$ has a $d$-packing of cycles $\mathcal{C}$ with $|\mathcal{C}|=p$.
%For any integer $r\ge 0$, a set $X\subseteq V(G)$ is a \defin{radius-$r$ hitting-set} of cycles in $G$ if $G-B_G(X,r)$ is a forest.  The \defin{radius-$r$ hi$\tau$$\tau$ing number} (of cycles in $G$), $\mathdefin{\tau_r(G)}$, is the minimum integer $t$ such that $G$ has a radius-$r$ hitting set of size $t$.


\begin{thm}\label{thm:main-in-intro}
  There exists a function $f:\mathbb{N}\to\mathbb{N}$ such that for all nonnegative integers $k$ and $d$,  for every graph $G$,  either $G$ contains a $d$-packing of $k$ cycles, or  there exists a subset $X$ of vertices of $G$ with $|X|\leq f(k)$ such that  $G-B_G(X,100d)$ is a forest.
\end{thm}


The \defin{length}, $\ell$, of a path $P:=x_0,\ldots,x_\ell$ is the number of edges in $P$.  For two vertices $x$ and $y$ in a  tree $T$, the \defin{$T$-path}, $\mathdefin{\pth_T(x,y)}$ is the unique path $x_0,\ldots,x_\ell$ in $T$ with $x_0=x$ and $x_\ell=y$. If $Y\subseteq V(T)\setminus\{x\}$ and $T[Y]$ is connected, then $\mathdefin{\pth_T(x,Y)}$ is the shortest path $x_0,\ldots,x_{\ell}$ in $T$ with $x_0=x$ and $x_\ell\in Y$.  For a graph $G$, and two vertices $x$ and $y$ of $G$, $\mathdefin{\dist_G(x,y)}$ is the length of a shortest path, in $G$, with endpoints $x$ and $y$.  For subsets $X$ and $Y$ of $V(G)$, $\mathdefin{\dist_G(X,Y)}:=\min\{\dist_G(x,y):(x,y)\in X\times Y\}$.  For an integer $r\ge 0$ and a vertex $x$ of $G$, $\mathdefin{B_G(x,r)}:=\{y\in V(G):\dist_G(x,y)\le r\}$.  For a subset $S$ of $V(G)$, $\mathdefin{B_G(S,r)}:=\bigcup_{x\in S}B_G(x,r)$, $\mathdefin{\diam_G(S)}:=\max\{\dist_G(x,y):\{x,y\}\in\binom{S}{2}\}$ and $\mathdefin{\radius_G(S)}$ is the minimum integer $r$ such that $S\subseteq B_G(v,r)$ for some $v\in V(G)$.

% \begin{comment}
% Note that $\tau_r$ and $\rho_d$ are not monotone graph parameters.  If $G'$ is a subgraph of $G$ then a distance-$d$ packing of cycles in $G'$ may not be a distance-$d$ packing of cycles in $G$ since $\dist_{G}(V(C),V(C'))$ may be less than $\dist_G(V(C),V(C'))$.  Similarly, a radius-$r$ hitting set $X$ for cycles in $G$ may not be a radius-$r$ hitting set for cycles in $G'$, since $B_{G'}(X,r)$ may be a strict subset of $B_{G}(x,r)\cap V(G')$.
%
% \begin{conj}
%   There exists functions $f:\N\to\N$ and $g:\N\to\N$ such that, for every graph $G$ and every $d,k\in\N$, $\rho_d(G) \ge k$ or $\tau_{g(d)}(G)\le f(k)$.
% \end{conj}
%
%
% \section{An Exponential Bound}
% \end{comment}

% \begin{lem}\label{far_away_er}
%   Let $G$ be a graph and let $C$ be a cycle in $G$ of length $\girth(G)$.  Then, for every $k,d\ge 0$ and $r\ge d$, at least one of the following is true:
%   \begin{compactenum}[(a)]
%     \item $G$ contains a cycle $C'$ with $\dist_G(V(C),V(C'))>r-d$;
%     \item $\rho_d(G)> k$; or
%     \item $\tau_{r+2d}(G) \le ck$.
%     \item $\tau_{5r-d}(G) \le 1$.
%   \end{compactenum}
% \end{lem}
%
% \begin{proof}
%   Assume that (a) does not hold. That is, every cycle $C'$ in $G$ has $\dist_G(V(C),V(C'))\le r-d$.  If $\girth(G) \le 4r$, then we can take any vertex $v\in V(C)$ and $B_G(v,5r-d)$ intersects every cycle in $G$, so $\tau_{5r-d}(G)\le 1$, in which case (d) holds.  Now assume $\girth(G) > 4r$.  If $G[B_G(V(C),r)]$ is not unicyclic, then $G[B_G(V(C),r)]$ contains a cycle of length at most
%   \[
%     \girth(G)/2 + 2r < \girth(G) - 2r + 2r = \girth(G) \enspace ,
%   \]
%   which is a contradiction.  Therefore $A:=G[B_G(V(C),r)]$ is unicyclic and $F:=G-B_G(V(C),r-d)$ is a forest.  Now apply the Hungarian result on $(A,F)$ to conclude that at least one of (b) or (c) holds.
% \end{proof}
%
% \begin{cor}\label{k_equals_one}
%   Let $G$ be a graph with $\rho_d(G)\le 1$.  Then $\tau_{3d}(G) \le 2c$ or $\tau_{4d}(G) \le 1$.
% \end{cor}
%
% \begin{proof}
%   If $G$ is a forest, then $\tau_{4d}(G)=0$ and there is nothing to prove. Otherwise, let $C$ be a cycle in $G$ of length $\girth(G)$ and apply \cref{far_away_er} with $r=d$ and $k=1$.  The fact that $\rho_d(G)\le 1$ rules out (a) and (b), leaving only (c) and (d).
% \end{proof}
%
%
%
% \begin{cor}
%   Let $G$ be a graph with $\rho_d(G)\le 2$.  Then $\tau_{?}(G) \le {?}$.
% \end{cor}
%
% \begin{proof}
%   Let $C$ be a cycle in $G$ of length $\girth(G)$.  Apply \cref{far_away_er} with $r={?}d$ and $k=2$. Then alternative (b) is not possible since, by assumption $\rho_d(G)\le 2$.  Alternatives (c) and (d) establish the corollary.  All that remains is to consider alternative (a).  In this case, $G$ contains a cycle $C'$ with $\dist_G(V(C),V(C'))>r$.  By \cref{short_or_unicycle_nearby}, $G$ contains a cycle $C''$ with $\dist_G(V(C),V(C''))\ge ??-d$
%
%
%
%   There are two cases to consider:
% \end{proof}
%
%
%
%
%
%

%\piotr{Maybe we could define a ball, e.g.\ $B_G(V(C),t)$ as a subgraph and not as subset of vertices? Maybe not.}

\piotr{Definitions}

Let $|P|$ be the \defin{length} of the path $P$, i.e.\ the number of edges in $P$.

Let $r$ be a positive integer and let $G$ be a graph.
A cycle $C$ in $G$ is \defin{$r$-unicyclic in $G$}
if $C$ is the only cycle in $G[B_G(V(C),r)]$.  A path $P$ in $G$ is \defin{$r$-acyclic in $G$} if $G[B_G(V(P),r)]$ is a tree.

\section{Tools}
For all positive integers $k$, put
\[
s(k):=\begin{cases}
4k(\log k + \log\log k +4)&\textrm{if $k\geq2$}\\
2&\textrm{if $k\leq1$.}
\end{cases}
\]
\begin{thm}
[\citet{Simonovits67}]\label{thm:simonovits}
Let $k$ be a positive integer and
let $G$ be a graph with all vertices of degree $2$ or $3$.
If $G$ contains at least $s(k)$ vertices of degree $3$, then
$G$ contains $k$ vertex-disjoint cycles.
\end{thm}

\piotr{The one below is under construction ...}
\begin{thm}[\citet{gyarfas.lehel:helly}]\label{thm:gyarfas-lehel}
   There exists a function $\ell:\N\times\N\to\N$ such that the following is true, for every $p,k\in\N$.
   Let $\mathcal{T}:=(T_1,\ldots,T_p)$ be a $p$-tuple of forests and let $\{\mathcal{A}_1,\ldots,\mathcal{A}_m\}$ be a set of $p$-subtrees of $\mathcal{T}$. Then either,
   \begin{compactenum}[(a)]
     \item $\{\mathcal{A}_1,\ldots,\mathcal{A}_m\}$ contains a set of $k$ pairwise-disjoint $p$-subtrees of $\mathcal{T}$; or
     \item there exists a $p$-subgraph $\mathcal{X}:=(X_1,\ldots,X_k)$ of $\mathcal{T}$ with $\sum_{i=1}^k|V(X_i)|\le \ell(p,k)$ such that $\mathcal{A}_i$ intersects $\mathcal{X}$ for each $i\in\{1,\ldots,m\}$.
   \end{compactenum}
\end{thm}


\section{The proof}



\begin{lem}\label{short_or_unicycle_nearby}
  Let $t$ be an integer with $t\ge 0$.
  Let $G$ be a graph and let $C$ be a cycle in $G$.
  Then $B_G(V(C),3t)$ contains a cycle $C'$ such that
  \begin{compactenum}[(a)]
    \item $B_G(V(C'),t)$ is unicyclic; or\label{short_or_unicycle_nearby:unicyclic}
    \item $C'$ has length at most $6t+2$.\label{short_or_unicycle_nearby:short}
  \end{compactenum}
\end{lem}

\begin{proof}
  Let $C_0:=C$.
  We construct inductively a sequence of pairs $(C_i,Q_i)_{i\geq0}$ such that
  (1) $C_i$ is a cycle in $G$;
  (2) $Q_i\subseteq C_i\cap C$, $Q_i$ is connected and contains at least one edge; and
  (3) $C_i$ contains at most $4t+1$ edges not in $Q_i$.
  Note that these two conditions imply that there is a path $P_i\subseteq C_i$  such that
  $P_i$ and $Q_i$ are edge-disjoint and $C_i=P_i\cup Q_i$.
  Thus, $P_i$ is a path of length at most $4t+1$ with both endpoints in $C$
  which implies that all vertices of $P_i$ are in distance at most $2t$ from $C$.
  In particular, $C_i\subseteq B_G(V(C),2t)$.

  Let $i\geq0$ and suppose that we already have defined $C_i$.
  If $B_G(V(C_i),t)$ is unicyclic, then $C_i$ witnesses~\eqref{short_or_unicycle_nearby:unicyclic}.

  Now suppose that $B_G(V(C_i),t)$ contains a cycle $D$ different than $C_i$.
  Let $u$ be a vertex of $D$ that is of maximal distance from $C_i$ among all vertices in $D$.
  Let $P(u)$ be a shortest path from $C$ to $u$.
  Since $u$ is incident to two vertices in $D$ there is a neighbor $v$ of $u$ in $D$ that is not in $P(u)$.
  Let $P(v)$ be a shortest path from $C_i$ to $v$.
  Note that both $P(u)$ and $P(v)$ are of length at most $t$.
  If $P(u)\cup P(v)\cup\{uv\}$ contains a cycle, say $E$, then $E$ has length at most $2t+1$ and $E\subseteq B_G(V(C_i),t)\subseteq B_G(V(C),3t)$, so $E$ witnesses~\eqref{short_or_unicycle_nearby:short}.
  Thus, we assume that $P:=P(u)\cup P(v)\cup\{uv\}$ contains no cycle so it must be a path.
  Also the length of $P$ is at most $2t+1$.

  Now there are three possibilities:
  \begin{compactenum}
    \item Both endpoints of $P$ are in $V(P_{i})$.
    In this case, $P\cup P_{i}$ contains a cycle of length at most
    $(2t+1)+(4t+1)=6t+2$ and this cycle is contained in $B_G(V(C_i),t)\subseteq B_G(V(C),3t)$, so it satisfies~\eqref{short_or_unicycle_nearby:short}.
    \item Both endpoints of $P$ are in $V(Q_{i})\setminus V(P_{i})$.
    In this case, we take $C_{i+1}$ to be a cycle in $Q_{i}\cup P$ and
    we take $Q_{i+1}:= Q_i\cap C_{i+1}$. % and $P_{i+1}:=P\cap C_{i+1}$.
    This works because $P$ has two distinct endpoints in $Q_{i+1}$
    and therefore $Q_{i+1}$ contains at least one edge,
    $Q_{i+1}\subseteq Q_i\subseteq C$, and
    $|P_{i+1}|\leq|P|\leq 2t+1$.
    Furthermore, note that $|Q_{i+1}| < |Q_{i}|$ because $C_{i+1}$ does not contain either endpoint of $Q_{i}$.
    \item Exactly one endpoint of $P$ is in $V(P_{i})$.
    In this case, $C_{i}\cup P$ has two cycles that each contain $P$.
    Each edge of $P_{i}$ belongs to exactly one of these two cycles.
    Therefore one of these cycles uses at most $\lfloor\frac{4t+1}{2}\rfloor=2t$ edges of $P_{i}$.
    We take $C_{i+1}$ to be this cycle and define
    $Q_{i+1}=C_{i+1}\cap Q_i$. %, and $P_{i+1}=C_{i+1}\cap (P\cup P_{i})$.
    Thus, $|P_{i+1}| \leq 2t+|P|\leq 4t+1$.
    Furthermore, note that $|Q_{i+1}| < |Q_{i}|$ because $C_{i+1}$ does not contain one of the endpoints of $Q_{i}$.
  \end{compactenum}
  This process eventually produces the desired cycle $C'$ from the statement since, at each step in the process $|Q_i|$ decreases.
\end{proof}

\begin{lem}\label{grow_unicycle}
  Let $d,R$ be integers such that $0\leq d \leq R$.
  Let $G$ be a graph and let $C$ be an $d$-unicyclic cycle in $G$.
  Then either:
  \begin{compactenum}[(a)]
    \item $G$ has a $d$-packing of $k$ cycles; or
    \label{grow_unicycle:item:packing}
    \item %\piotr{simplest possible version}
    there exists $Y\subseteq V(G)\setminus B_G(V(C),d)$ and $X\subseteq V(G)$ of size $\Oh(k\log k)$ such that $Y\subseteq B_G(X,2R+d)$ and
    $C$ is $R$-unicyclic in $G-Y$.
  \end{compactenum}
\end{lem}

\begin{proof}
  Let $G_{R}:=G[B_G(V(C),R)]$.
  For each $v\in G_{R}\setminus V(C)$, select a vertex $p_v$ (the \defin{BFS parent} of $v$) such that $\dist_G(p_v,V(C))=\dist_G(v,V(C))-1$.
  Let $U$ (a \defin{BFS unicycle}) be the subgraph of $G_{R}$ with vertex set $V(U):=V(G_{R})$ and edge set $E(U):=E(C)\cup\{vp_v:v\in V(U)\setminus V(C)\}$.  Let $v_0$ be an arbitrary vertex of $C$.

  Let $e\in E(G_{R})\setminus E(U)$.
  We define $C_e$ to be the unique cycle in $U\cup \set{e}$ that does not contain $v_0$.
  If $C_e$ has no edges in $C$ then we say that $e$ is \defin{$C$-null}.
  Otherwise, we say that $e$ is \defin{$C$-nonnull}.
  Let $P_{e}$ be the path or cycle formed by the edges in $E(C_{e})\setminus E(C)$ and let $Q_{vw}$ be the (possibly empty) path formed by the edges in $E(C_{e})\cap E(C)$.
  Consider the auxiliary graph $H$ with the vertex-set $\set{e\mid e\in E(G_{R})\setminus E(U)}$ and two distinct elements $e$ and $e'$ are adjacent in $H$ if $\dist_G(V(P_{e}),V(P_{e'})) \le d$.  Let $I$ be a maximal independent set in $H$.

  Suppose first that $|I|\ge k+4k\log k$.
  Therefore,
  either (1) $I$ contains at least $k$ $C$-null edges;
  or (2) $I$ contains at least $4k\log k$ $C$-nonnull edges.

  In case (1), let $J$ be the set of $C$-null edges in $I$.
  Consider two distinct $e,e'\in J$.
  Since both edges are $C$-null we have $P_e=C_e$ and $P_{e'}=C_{e'}$.
  Now since both edges are in $I$ we have
  \[
  \dist_G(V(C_e),V(C_{e'}))=\dist_G(V(P_e),V(P_{e'}))>d.
  \]
  Thus, $\set{C_e \mid e\in J}$ is a $d$-packing of at least $k$ cycles in $G$ and \eqref{grow_unicycle:item:packing} holds.

  In case (2), let $J$ be the set of $C$-nonnull edges in $I$.
  Let $G'$ be the graph obtained from $C$ by adding $P_e$ for each $e\in J$.
  Since $\dist_G(V(P_e),V(P_{e'}))>d\ge0$, $G'$ contains only vertices of degree $2$ and degree $3$, and the degree-$3$ vertices correspond to the endpoints of paths in $\set{P_e \mid e\in J}$.
  Therefore, $G'$ contains $2|J|\geq 8k\log k$ vertices of degree $3$.
  By \cref{thm:simonovits}, $G'$ contains a set $\mathcal{D}$ of $k$ pairwise vertex-disjoint cycles.
  We claim that $\mathcal{D}$ is a $d$-packing in $G$.
  Let $D$ and $D'$ be two distinct cycles in $\mathcal{D}$.
  Let $v$ and $v'$ be vertices of $D$ and $D'$, respectively,
  such that $\dist_G(v,v')=\dist_G(V(D),V(D'))$.
  If $v\in V(D)\setminus V(C)$ and $v'\in V(D')\setminus V(C)$, then
  $v$ lies $P_e$ and $v'$ lies in $P_{e'}$ for some  $e,e' \in I$.
  Since $D$ and $D'$ are vertex-disjoint $e$ and $e'$ are distinct.
  Since $e,e'\in I$, we have $\dist_G(v,v') \geq \dist_G(V(P_e),V(P_{e'}))>d$, as desired.
  Thus, without lost of generality we assume that $v \in V(C)$.
  Let $P$ be a shortest path from $v$ to $v'$ in $G$.
  If $P$ contains a vertex outside $G_r$, then
  $P$ starts at $v$ in $C$ and takes at least $d+1$ edges to leave $B_G(V(C),d)$ and then takes at least one edge to return to $B_G(V(C),d)$.
  Thus, $|P|\geq d+2 > d$ in this case, as desired.
  Otherwise, the path $P$ is contained in $G_d$.
  Recall that $C$ is $d$-unicyclic in $G$.
  Thus, $P$ consists of a segment $P'$ contained in $C$ and a segment $P''$ which is a shortest path from $C$ to $v'$.
  Suppose now that $v'$ does not lie on $C$.
  In this case $v' \in V(P_{e'})$ for some $e'\in J$.
  Then $P_{e'}\subseteq D'$ and in particular both endpoints of $P_{e'}$ lie in $D'$.
  Since $P$ is a shortest path between $V(D)$ and $V(D')$,
  we conclude that $v'$ is an endpoint of $P_{e'}$ and in particular $v'$ lies in $C$, a contradiction.
  Thus, we have that both $v$ and $v'$ lie in $C$.
  Since $P$ is a shortest path between $V(D)$ and $V(D')$, no edge of $P$ is contained in $D$. Note also that $P$ contains at least one edge as $D$ and $D'$ are vertex-disjoint. Therefore, the first edge of $P$ is an edge of $C$ and $v$ is the endpoint of a path $P_e$ that is contained in $D$. Similarly, $v'$ is the endpoint of a path $P_{e'}$ that is contained in $D'$. Therefore $\dist_G(v,v')\ge \dist_G(V(P_e),V(P_{e'}))>d$, as desired.
  Thus indeed $\mathcal{D}$ a $d$-packing of  $k$ cycles in $G$ and~\eqref{grow_unicycle:item:packing} holds.

  It remains to consider the case when $|I| < k+4k\log k$.
  Since $I$ is a maximal independent set in $H$, it is a dominating set in $H$: Every vertex of $H$ is either in $I$ or adjacent to a vertex in $I$.  In $G$ this corresponds to the fact that, for every edge of $e=uv\in E(G_R)\setminus E(U)$ there exists $e'=u'v'\in I$ such that $\dist_G(V(P_{e}),V(P_{e'}))\le d$.
  Since $\dist_G(u',V(C))\leq R$ and $\dist_G(v',V(C))\leq R$,
  we have $V(P_{e'}) \subseteq B_G(\set{u',v'},R)$. Since $\dist_G(V(P_e),V(P_{e'}))\le d$,  $B_G(\set{u',v'},R+d)$ contains a vertex of $P_e$.
  Finally, $\dist_G(u,V(C))\leq R$ and $\dist_G(v,V(C))\leq R$ implies that
  $B_{G}(\set{u',v'},R+d+R)$ contains at least one of $u$ or $v$.

  Let $X\subseteq V(G)$ be the set obtained by taking both endpoints of $e$ for each $e\in I$.  Then, for each $e\in E(G_R)\setminus E(U)$, $B_G(X,2R+d)$ contains at least one endpoint of $e$, which we place in the set $Y$.  Then $C$ is $R$-unicyclic in $G-Y$ and $Y\subseteq B_G(X,2R+d)$, as required.
\end{proof}


\begin{lem}\label{double_unicycle}
  Let $R,d$ be integers with $R\geq 2d\geq 0$.
  Let $G$ be a graph,
  let $C_1$ and $C_2$ be cycles in $G$ such that
  $\dist_G(V(C_1),V(C_2))>2d$.
  Let $Y_i\subseteq V(G) - B_G(V(C_i),d)$ be such that $C_i$ is $R$-unicyclic cycle in $G-Y_i$,
  for each $i\in[2]$.
  Then either:
  \begin{compactenum}[(a)]
    \item $G$ has a $d$-packing of $k$ cycles; or
    \item there exists $X\subseteq V(G)$ such that \[
    B_{G-Y_1}(V(C_1),R)\cap B_{G-Y_2}(V(C_2),R) \subseteq B_G(X,2R+d) \cup B_G(Y_1\cup Y_2,R).
    \]
  \end{compactenum}
\end{lem}

\begin{proof}
    Let $i\in[2]$.
    Since $C_i$ is $R$-unicyclic in $G-Y_i$ and $Y_i\cap B_G(V(C_i),d) = \emptyset$, we conclude that $C_i$ is $d$-unicyclic in $G$.
    Let $U_i = G[B_{G-Y_i}(V(C_i),R)]$.
    For each vertex $v$ in $U_i$ let $P_{i,v}$ be the shortest path from $v$ to $V(C_i)$ in $U_i$.

    Let $v$ be in $V(U_1)\cap V(U_2)$.
    Define $Q_v:= P_{1,v}\cup P_{2,v}$.
    Let $W=\set{v \in V(U_1)\cap V(U_2) \mid V(Q_v)\cap (Y_1\cup Y_2) =\emptyset}$.
    Let $w \in W$.
    Define $P_w$ to be a shortest path from $V(C_1)$ to $V(C_2)$ in $Q_w$.
    Note that $P_w$ is a path between $V(C_1)$ and $V(C_2)$ in $G-(Y_1\cup Y_2)$.

%    Note that for every $v \in W$, for each $i\in[2]$,
%    $P_v \cap U_i=P_v\cap G[V(U_i)]$.
    %Note that $P_v$ consists of two segments:
    %$u_1P_vv$ and $vP_vw_2$, where $u_i\in V(C_i)$ %for each $i\in[2]$ and each segment is ...

    Let $H$ be an auxiliary graph with the vertex set $W$ and such that two distinct vertices $w$ and $w'$ are adjacent in $H$ if $\dist_G(V(P_w),V(P_{w'}))\leq d$.
    Let $I$ be a maximal independent set in $H$.

    Suppose that $|I|\geq 4k\log k$.
    Let $G'$ be the subgraph of $G$ obtained from $C_1\cup C_2$ by adding $P_w$ for each $w\in I$.
    Since $\dist_G(V(P_w),V(P_{w'}))>d\ge0$ for all distinct $w,w'\in I$, $G'$ contains only vertices of degree $2$ and degree $3$, and the degree-$3$ vertices correspond to the endpoints of paths in $\set{P_w\mid w\in I}$.
    Therefore, $G'$ contains $2|I|\geq 8k\log k$ vertices of degree $3$.
    By \cref{thm:simonovits}, $G'$ contains a set $\mathcal{D}$ of $k$ pairwise vertex-disjoint cycles.

    We claim that $\mathcal{D}$ is a $d$-packing in $G$.
    Let $D$ and $D'$ be two distinct cycles in $\mathcal{D}$.
    Let $w$ and $w'$ be vertices of $D$ and $D'$, respectively,
    such that $\dist_G(w,w')=\dist_G(V(D),V(D'))$.
    Suppose first that $w\in V(P_v)$ and $w'\in V(P_{v'})$ for some  $v,v' \in I$.
    Since $D$ and $D'$ are vertex-disjoint, $v$ and $v'$ are distinct.
    Since $v,v'\in I$, we have $\dist_G(w,w') \geq \dist_G(V(P_v),V(P_{v'}))>d$, as desired.

    Now suppose that $w, w' \in V(C_i)$ for some $i\in[2]$.
    Consider a shortest path $P$ from $w$ to $w'$ in $G$.
    Recall that $C_i$ is $d$-unicyclic in $G$.
    Thus, if $P$ contains a vertex not in $C_i$, then $P$ contains a vertex outside $B_G(V(C_i),d)$ and so $\dist_G(w,w')=|P|>2d+2$.
    Now suppose that $P\subseteq C_i$.
    Since $P$ is a shortest path between $V(D)$ and $V(D')$, no edge of $P$ lies in $D\cup D'$.
    Therefore $G'$ contains two edges of $D$ incident to $w$ and a third edge in $P$ (and in $C_i$) incident to $w$.
    Therefore $w$ has degree $3$ in $G'$, so $w\in V(P_{v})$ and $P_v\subseteq D$.
    In particular, $v\in I$.
    By the same argument, $w'\in V(P_{v'})$ and $v'\in I$.
    Since $D$ and $D'$ are vertex-disjoint $v\neq v'$.
    Therefore, $\dist(v,w)\ge \dist(V(P_v),V(P_{v'}))>d$.


    Now suppose that $w\in V(C_1)$ and $w'\in V(C_2)$.
    In this case we simply have that
    $\dist_G(V(D),V(D'))=\dist_G(w,w')\geq \dist_G(V(C_1),V(C_2))>2d\geq d$, as desired.
    (The case that $w\in V(C_1)$ and $w'\in V(C_2)$ is completely symmetric.)

    It remains to consider the case that
    $w\in V(C_1)$ and $w'\in V(P_{v'})- (V(C_1)\cup V(C_2))$ for some $v'\in I$.  Let $P'$ be the maximal subpath of $P_{v'}$ that contains the endpoint of $P_{v'}$ in $V(C_1)$ and that is contained in $B_{G-Y_1}(V(C_1),R)$. Let $P''$ be the subpath of $P_{v'}$ formed by the edges of $P_{v'}$ not in $P'$.  ($P''$ may have no edges.)

    If $w'\in V(P')$ then let $P$ be a shortest path in $G$ from $w$ to $w'$.  If $P$ contains a vertex not in $B_G(V(C_1),d)$ then $|P|> d$.
    Now suppose that $V(P)\subseteq B_G(V(C_1),d)$.
    Since $C_1$ is $d$-unicyclic in $G$, $P$ consists of a segment of $C_1$ followed by a segment of $P'$.
    Since $P$ is a shortest path between $V(D)$ and $V(D')$, no edge of $P$ lies in $D\cup D'$.  Therefore, $w$ is incident to two edges of $D$ and a third edge of $P$ in $C_1$.  Therefore $w\in V(P_v)$ for some $v\in I$ and $P_v\subseteq D$.  Since $D$ and $D'$ are vertex-disjoint, $v\neq v'$.  Therefore $\dist_G(w,w')\ge\dist_G(V(P_v),V(P_{v'}))>d$.

    Now assume $w'\in V(P'')$ and suppose, for the sake of contradiction that $\dist_G(w,w')\le d$. Therefore $\dist_G(V(C_1),w')\le d$.  Then $P''$ begins at a vertex $x$ with $\dist_G(x,V(C_1))\ge R+1$, then proceeds to $w'$ using at least $R+1-d$ edges and then proceeds to $C_2$ using at least $\dist_G(V(C_1),V(C_2))-\dist_G(V(C_1),w')\ge 2d+1-d=d+1$ edges.  Therefore $P''$ has at least $R+2$ edges.  Therefore $P_v$ has length at least $2R+3$. This is a contradiction because $P_v\subseteq Q_v$ and $Q_v$ has at most $2R$ edges.

    Now suppose that $|I|<4k\log k$.
    Since $I$ is a maximal independent set in $H$, the set $I$ is also a dominating set:
    For every $w\in W$ either $w\in I$ or $w$ has a neighbor in $I$.

    Let $X:=\set{x \in V(G) \mid \textrm{$x$ is an endpoint of $P_v$ for some $v\in I$}}$.
    We claim that
    $B_{G-Y_1}(V(C_1),R)\cap B_{G-Y_2}(V(C_2),R)\subseteq B_G(X\cup Y_1\cup Y_2,R)$.

    Consider a vertex $v \in B_{G-Y_1}(V(C_1),R)\cap B_{G-Y_2}(V(C_2),R)$.
    We split into two cases: $v\not\in W$ and $v\in W$.
    If $v\not\in W$, then $Q_v$ contains a vertex $y \in Y_1\cup Y_2$.
    Then $\dist_G(y,v) \leq R$ and therefore
    $v\in B_G(Y_1\cup Y_2,R)$, as desired.
    If $v\in W$, then we again split into two cases: $v\in I$ and $v\notin I$.
    If $v\in I$, then both endpoints of $P_v$ are in $X$ so $v\in B_G(X,R)$.
    If $v\notin I$, then there exists some $v'\in I$ such that $\dist_G(V(P_v),V(P_{v'}))\le d$.
    Both endpoints of $P_{v'}$ are in $X$, so $B_G(X,R)$ contains $V(P_{v'})$.  Then $B_G(X,R+d)$ contains some vertex of $P_v$.  Since $\dist_G(v,x)\le R$ for each $x\in V(P_v)$, $B_G(X,2R+d)$ contains $v$.
\end{proof}

\begin{thm}\label{thm:the-big-ball-of-wax}
  Let $f$ and $g$ be the following functions:
  \[
    f(x)= x^{x^x}\cdot x\log x,\qquad
    g(x)= 15x +3.
  \]
  For all non-negative integers $d$ and $k$, for every graph $G$, either $G$ contains a $d$-packing of $k$ cycles or there exists $X\subseteq V(G)$ with $|X|\leq f(k)$ such that $G-B_G(X,g(d))$ is a forest.
\end{thm}

\begin{proof}
  Let $d$ and $k$ be non-negative integers. Let $G$ be a graph. If $G$ contains a $d$-packing of $k$ cycles then there is nothing to prove. Thus, assume the opposite. Let $\set{\mathdefin{C_1,\ldots,C_p}}$ be a maximal $2d$-packing of cycles that are $d$-unicyclic in $G$.  Since $G$ has no $d$-packing of $k$ cycles, $p<k$.

  Begin with every vertex of $G$ \defin{unmarked}.  While $G$ contains a cycle $D$ having no marked vertices and no vertex in $B_G(\bigcup_{i=1}^p V(C_i),5d)$, do the following:  Apply \cref{short_or_unicycle_nearby} to $D$ to find a cycle $D'$ with all vertices in  $B_G(V(D),3d)$ such that either $|D'|\leq 6d+2$ or $D'$ is $d$-unicyclic in $G$. Since $V(D)\cap B_G(\bigcup_{i=1}^p V(C_i),5d)=\emptyset$ and $V(D')\subseteq B_G(V(D),3d)$, we have that $V(D')\cap B_G(\bigcup_{i=1}^p V(C_i),2d)=\emptyset$. Since $\{C_1,\ldots,C_p\}$ is maximal, $D'$ is not $d$-unicyclic in $G$, so $D'$ has length at most $6d+2$.  \defin{Mark} all vertices in $B_G(V(D'),d)$. This completes the description of the process.  Let $\mathdefin{M}$ be the set of vertices of $G$ marked by the process.  The set $\{D_1,\ldots,D_{q}\}$ of cycles found by this process is a $d$-packing of cycles in $G$.  Since $G$ has no $d$-packing of size $k$, $q<d$. Let $\mathdefin{X_M}:=\{x_1,\ldots,x_{q}\}$ where  $x_i$ is a vertex of $D_i$ for each $i\in\{1,\ldots,q\}$. Then $M\subseteq B_G(X_M,4d+1)$.

Every cycle in $G-M$ contains a vertex in $\bigcup_{i=1}^p B_G(V(C_i),5d)$.
Let $\mathdefin{R}:=7d$.
Let
\[
  \textstyle\mathdefin{F_0}:=G-\left(M\cup \bigcup_{i=1}^p B_G(V(C_i),R-2d)\right)
\]
and let
\[
  \textstyle\mathdefin{F^-_0}:=G-\left(M\cup \bigcup_{i=1}^p B_G(V(C_i),R-d)\right) \enspace .
\]
Then $F_0$ is a forest and $F^-_0$  (which is an induced subgraph of $F_0$) is also a forest.

For each $i\in\{1,\ldots,p\}$, apply \cref{grow_unicycle} to obtain a set $\mathdefin{Y_i}\subseteq V(G)\setminus B_G(V(C_i),d)$ and a set $\mathdefin{X_i}\subseteq V(G)$ of size $\Oh(k\log k)$ such that $Y_i\subseteq B_G(X_i,2R+d)$ and such that, for each $i\in\{1,\ldots,p\}$, $C_i$ is $R$-unicyclic in $G-Y_i$.  For each $i\in\{1,\ldots,p\}$, let $U_i$ be the component of $G[B_{G-Y_i}(V(C_i),R)]$ that contains $V(C_i)$ and let $U_i^-:=U_i[B_{G-Y_i}(V(C_i),R-d)]$.  Then each of $U_i$ and $U^-_i$ is a unicyclic graph whose only cycle is $C_i$.   For each $i\in\{1,\ldots,p\}$, let $y_i$ be an arbitrary vertex of $C_i$, let $F_i:=U_i-y_i$, and let $F^-_i:=U^-_i-y_i$. Then $F_i$ and $F_i^-$ are forests, for each $i\in\{1,\ldots,p\}$.

For each $(i,j)$ with $1\le i < j \le p$, let $\mathdefin{Y_{i,j}}:=B_{G-Y_i}(V(C_i),R)\cap B_{G-Y_j}(V(C_j),R)$.  Apply \cref{double_unicycle} to $C_i$ and $C_j$ to obtain a subset $\mathdefin{X_{i,j}}$ of $V(G)$ such that $Y_{i,j}\subseteq B_G(X_{i,j},2R+d) \cup B_G(Y_i\cup Y_j,R)$.

% Let
% \[
%   \textstyle\mathdefin{F^-}:=G-\left(M\cup \bigcup_{i=1}^p B_G(V(C_i),R-d)\right) \enspace .
% \]
% Then $F^-$ is a forest, since $F^-\subseteq F$.

% \pat{We need to rework this definition so that we can have admissible tuples of the form $(e,i,e,j)$ with $i\neq j$. Currently, the forest leg of such a tuple has no vertices, but we still need to hit it. (Think of a cycle $y_1,\ldots,y_p$ where $y_i$ is in $B_G(V(C_i),R)$ and not in any other ball.) I think this means that we want to define the forest leg of $(e,i,e',j)$ as $eP_0e'$.}
Let $i,j\in[p]$. An \defin{$i$-exit edge} is an edge $e$ of $U_i$ with exactly one endpoint in $U^-_i$.
A $4$-tuple $(e,i,e',j)$ is a \defin{good tuple} if
\begin{compactitem}
  \item $e\neq e'$ and $e$ and $e'$ are incident to the same component of $F^-$;
  \item $e$ is an $i$-exit edge;
  and \item $e'$ is a $j$-exit edge.
\end{compactitem}
Each good tuple $t:=(e,i,e',j)$  defines a walk $\mathdefin{W_t}:=P_1eP_0e'_2$ in $G$ where
\begin{compactitem}
  \item $P_1$ is the shortest path in $U_i^-$ from $V(C_i)$ to the end point of $e$ in $U^-_i$;
  \item $P_0$ is the unique path in $F_0^-$ from the endpoint of $e$ in $F^-_0$ to the endpoint of $e'$ in $F^-_0$; and
  \item $P_2$ is the shortest path in $U_j^-$ from the endpoint of $e'$ in $U_j^-$ to $V(C_j)$.
\end{compactitem}
We call the path $P_0$ the \defin{forest leg} of $W_t$, $P_1$ the \defin{first leg} of $W_t$, and $P_2$ the \defin{second leg} of $W_t$.  The walk $eP_0e'$ is called the \defin{extended forest leg} of $W_t$.


% We distinguish the following segments of $W-\set{e,e'}$:
% \begin{compactenum}[(a)]
% \item \defin{first leg}: the subpath of $P_1$ that contains the first $5d$ edges of $P_1$;
% \item \defin{first buffer}: the subpath of $P_1$ that contains the last $d$ edges of $P_1$;
% \item \defin{forest leg}: $P_0$;
% \item \defin{second buffer}: the subpath of $P_2$ that contains the first $d$ edges of $P_2$; and
% \item \defin{second leg}: the subpath of $P_2$ that contains the last $5d$ edges of $P_2$.
% \end{compactenum}


% Let
% \[
% \textstyle M':= M \cup \bigcup_{i\in[p]} (Y_i\cup\{v_i\}) \cup \bigcup_{i,j\in[p],\ i<j} Y_{i,j}).
% \]
% \pat{Why the awkward double-indexing in the second union? I suggest
% For each $i\in\{1,\ldots,p\}$, let $\mathdefin{y_i}$ be an arbitrary vertex of $C_i$ and define
Define
\[
  \textstyle \mathdefin{M'}:= M \cup \bigcup_{i=1}^p \left(Y_i\cup\{y_i\} \cup \bigcup_{j=i+1}^{p} Y_{i,j}\right).
\]
A good tuple $t$ is \defin{admissible} if $B_G(V(W_t),d) \cap M' = \emptyset$.


\begin{clm}\label{grow_in_forests}
  Let $t:=(e,i,e',j)$ be a good admissible tuple where $P_1eP_0e'P_2:=W_t$.  Then
  \begin{compactenum}[(i)]
    \item $V(P_1)\cap V(P_0)=\emptyset$ and $V(P_2)\cap V(P_0)=\emptyset$;\label[p]{gif_disjoint}
    \item $V(P_1)\cap B_G(V(C_\ell),R)=\emptyset$ for each $\ell\in\{1,\ldots,p\}\setminus\{i\}$ and $V(P_2)\cap B_G(V(C_\ell),R)=\emptyset$ for each $\ell\in\{1,\ldots,p\}\setminus\{j\}$;\label[p]{gif_unique_ends}
    \item $G[B_G(V(P_0),d)]=F_0[B_{F_0}(V(P_0),d)]$; \label[p]{gif_zero}
    \item $G[B_G(V(P_1),d)]=F_i[B_{F_i}(V(P_1),d)]$ and $G[B_G(V(P_2),d)]=F_j[B_{F_j}(V(P_2),d)]$; \label[p]{gif_ij}
  \end{compactenum}
\end{clm}

\begin{clmproof}\
  \begin{compactenum}[(i)]
    \item It follows immediately from their definitions that $U_i^-$ and $F_0^-$ are vertex disjoint, for each $i\in\{1,\ldots,p\}$. Thus (i) follows immediately since $P_1\subseteq U^-_i$, $P_2\subseteq U^-_j$, and $P_0\subseteq F_0^-$.

    \item If $V(P_1)\cap B_G(V(C_\ell),R)$ contains a vertex $v$ for some $\ell\neq i$ then $v\in Y_{i,\ell}$ (if $i<\ell$) or $v\in Y_{\ell,i}$ (if $\ell < i$).  In either case $t$ would be inadmissible.  The same argument applies if $V(P_2)\cap B_G(V(C_\ell),R)$ contains a vertex $v$ for some $\ell\neq j$.

    \item Since $t$ is admissible, $B_G(V(P_0),d)$ contains no vertex in $M$.  Since $P_0\subseteq F^-_0$, $B_G(V(P_0),d)$ contains no vertex in $B_G(V(C_i),R-2d)$ for each $i\in\{1,\ldots,p\}$.  Therefore $B_G(V(P_0),d)\subseteq V(F_0)$.  Since $F_0$ is an induced subgraph of $G$,  $B_G(V(P_0),d)=B_{F_0}(V(P_0),d)$.  Again, since $F$ is an induced subgraph of $G$, $G[B_G(V(P_0),d)]=F_0[B_{F_0}(V(P_0),d)]$.

    \item Since $t$ is admissible, $B_G(V(P_1),d)$ contains no vertex in $Y_i$.  Therefore,
    $B_G(V(P_1),d)=B_{G-Y_i}(V(P_1),d)$. Since $P_1$ contains a vertex of $C_i$, $P_1$ is contained in $U_i$ and $B_G(V(P_1),d)=B_{U_i}(V(P_1),d)$.  Since $t$ admissible, $B_G(V(P_1),d)$ does not contain $y_i$.  Since $F_i$ is an induced subgraph of $G$, $B_G(V(P_1),d)=B_{F_i}(V(P_1),d)$.  Since $F_i$ is an induced subgraph of $G$, $G[B_G(V(P_1),d)]=F_i[B_{F_i}(V(P_1),d)]$.  The same argument, using $j$ in place of $i$ and $P_2$ in place of $P_1$ establishes that $G[B_G(V(P_2),d)]=F_j[B_{F_j}(V(P_2),d)]$. \qedhere
  \end{compactenum}
\end{clmproof}


The following lemma allows us to finish the proof by finding a set of balls that intersects the forest leg of each admissible tuple.

\begin{clm}\label{hit_cycle}
  Let $C$ be a cycle in $G$.  Then  $C$ contains a vertex in $B_G(M',R+d)$ or $C$ contains the extended forest leg of some admissible tuple.
\end{clm}

\begin{clmproof}
  Since $F_0^-$ is a forest, $C$ must contain some vertex not in $F_0^-$.  Therefore $C$ contains a vertex in $F_i$ for some $i\in\{1,\ldots,p\}$.

  Since $F_i$ is an induced forest in $G$, $C$ contains a vertex not in $F_i$. If $C$ contains a vertex in $Y_i\cup\{y_i\}$ then we are done, since $Y_i\cup\{y_i\}\subseteq M'\subseteq B_G(M',R+d)$.

  Therefore, $C$ contains an $i$-exit edge $v_0v_1$ with $v_0\in V(F^-_i)$ and $v_1\not\in V(F^-_i)$ for some $i\in\{1,\ldots,p\}$.  If $v_0\in V(F^-_j)$ for some $j\in\{1,\ldots,p\}\setminus\{i\}$, then $v_0\in Y_{i,j}$ (if $i< j$) or $v_0\in Y_{j,i}$ (if $j<i$).  In either case, $v_0\in M'\subseteq B_G(M',d)$.  We now assume that $v_0\notin F^-_j$ for any $j\in\{1,\ldots,p\}$.  Therefore $v_1$ is a vertex of $F_0^-$.

  Consider the maximal path $v_1,\ldots,v_{r-1}$ in $C$ that contains $v_1$ and that is contained in $F_0^-$. Let $v_r$ be the neighbour of $v_{r-1}$ in $C-v_{r-2}$.  (Possibly $v_r=v_0$)  Since $v_r$ is not a vertex of $F_0^-$, $v_r\in V(F^-_j)$ for some $j\in\{1,\ldots,p\}$.  Then $v_0,\ldots,v_{r}$ is the extended forest leg of the tuple $t:=(v_0v_1,i,v_{r-1}v_r,j)$.   If $t$ is admissible, then there is nothing more to prove.  If $t$ is inadmissible, then the forest leg of $W_t$ contains a vertex in $B_G(M',d)\subseteq B_G(M',R+d)$ or the first or second leg of $t$ contains a vertex in $B_G(M',d)$.  If the forest leg $v_1,\ldots,v_{r-1}$ of $t$ contains a vertex in $B_G(M',d)$ then $C\supseteq v_1,\ldots,v_{r-1}$ contains a vertex in $B_G(M',d)\subseteq B_G(M',R+d)$.  If the first leg of $t$ contains a vertex in $B_G(M',d)$ then $v_0\in B_G(M',R+d)$.  If the second leg of $t$ contains a vertex in $B_G(M',d)$ then $v_r\in B_G(M',R+d)$.
\end{clmproof}

At this point it is tempting to mimic the proofs of \cref{grow_unicycle} and \cref{double_unicycle} and define an auxilliary graph $H$ with vertex set $V(H):=\{t:\textrm{$t$ is an admissible good tuple}\}$ and that contains an edge $st$ if and only if $\dist_G(V(W_s),V(W_t))\le d$. However, this does not work because $W_s$ and $W_t$ do not have diameter bounded by any function of $d$.  More precisely this fails because the assumption that $B_G(X,r)$ contains a vertex in the (extended) forest leg of $t$ does not imply that $B_G(X,r+g(d))$ contains a vertex in the (extended) forest leg of $s$, for any function $g$.  Instead we have to resort to the Hungarian Lemma, which is what the next paragraph is preparing for.

% \begin{comment}
% %-is called an \defin{ear}.
% The \defin{first leg} of $(e,i,e',j)$ is the shortest path in $G[B_{G-Y_i}(V(C_i),R)]$ from $V(C_i)$ to $e$.  The \defin{second leg} of $(e,i,e',j)$ is the shortest path in $G[B_{G-Y_j}(V(C_j),R)]$ from $V(C_j)$ to $e'$.  The \defin{forest leg} of $(e,i,e',j)$ is the unique path in $F^-$ from $e$ to $e'$.

% Let $(e,i,e',j)$ be a good tuple with first leg $P_1$, second leg $P_2$, and forest leg $P_0$.
% The \defin{thickened first leg} of $(e,i,e',j)$ is $B_G(V(P_1)\cap B_G(V(C_i),R-d),d)$.
% The \defin{thickened second leg} of $(e,i,e',j)$ is $B_G(V(P_2)\cap B_G(V(C_j),R-d),d)$.
% The \defin{thickened forest leg} of $(e,i,e',j)$ is $B_G(V(P_0),,d)$.


% % The path $P$ from an endpoint of $e$ to an endpoint of $e'$ in $F^-$ is called an \defin{$(e,e')$-ear}.  The \defin{extension} $P^+$ of $P$ is the path obtained by concatenating the shortest path in $G$ from $V(C_i)$ to $e$ in $G[B_G(V(C_i),R)]-Y_i$, the edge $e$, the path $P$, the edge $e'$ and the shortest path from $e'$ to $V(C_j)$ in $G[B_G(V(C_j),R]-Y_j$.



% The \defin{forest thickening} of $P$ is $\thick_F(P) := B_G(V(P_0)\cap V(F^-),d)$.  The \defin{first thickening} of $(e,i,e',j)$ is $\thick_i(P):=B_G(V(P_1)\cap B_G(V(C_i),R-d),d)$.
% The \defin{$j$-thickening} of $(e,i,e',j)$ is $\thick_i(P):=B_G(V(P_2)\cap B_G(V(C_j),R-d),d)$.

% A good tuple $(e,i,e',j)$ is \defin{admissible} if its extension $P^+$ contains no vertex in $B_G(M\cup \bigcup_{i=1}^p (Y_i\cup\{v_i\}\cup \bigcup_{j=i+1}^p Y_{i,j}),d)$.
% It follows from this definition that, if $P$ is admissible, then
% \begin{compactitem}
%   \item $\thick_F(P) = B_F(V(P)\cap V(F^-),d)$,
%   \item $\thick_i(P) = B_{G-(Y_i\cup\{v_i\})}(V(P^+)\cap B_G(V(C_i),R-d),d)$, and
%   \item $\thick_j(P^+) = B_{G-(Y_j\cup\{v_j\})}(V(P^+)\cap B_G(V(C_j),R-d),d)$.
% \end{compactitem}
% \end{comment}

Let $t=(e,i,e',j)$ be an admissible good tuple with $P_1eP_0e'P_2:=W_t$.
Define the $(p+1)$-tuple $\mathdefin{\Psi(t)}:=(\Psi_0(t),\Psi_1(t),\ldots,\Psi_p(t))$ where
\begin{align*}
% \Psi_0(t) &:= B_G(V(Q_0),d),\\
\mathdefin{\Psi_\ell(t)} &:=\begin{cases}
B_{F_0}(V(P_0),d)&\textrm{if $\ell=0$,}\\
B_{F_i}(V(P_1),d)&\textrm{if $\ell=i$ and $\ell\neq j$,}\\
B_{F_j}(V(P_2),d)&\textrm{if $\ell\neq i$ and $\ell= j$,}\\
B_{F_i}(V(P_1),d)\cup B_{F_j}(V(P_2),d)&\textrm{if $\ell= i$ and $\ell= j$,}\\
\emptyset&\textrm{if $\ell\notin\{0,i,j\}$.}
\end{cases}
\end{align*}
Note that, by \cref{grow_in_forests}\cref{gif_zero} and \cref{grow_in_forests}\cref{gif_ij}, each of the balls that defines $\Psi$ is equivalent to a ball in $G$.

\begin{clm}
Let $t$ be an admissible good tuple. Then
\begin{compactenum}[(i)]
\item $G[\Psi_0(t)]$ is a connected subgraph of $F_0$,
% \piotr{discuss definition of $F$}\pat{Why?} \piotr{We don't remove the $Y$-sets in the defintion of $F$.}
\item For each $\ell\in\{1,\ldots,p\}$, $G[\Psi_\ell(t)]$ is a subgraph of $F_i$ that has at most two components.
%$G[B_G(V(C_\ell),R)]$

% and consists of at most two components.
\end{compactenum}
\end{clm}

\begin{clmproof}\
  \begin{enumerate}[(i)]
    \item Since $P_0$ is a connected subgraph of $F_0$, $G[\Psi_0(t)]=F_0[\Psi_0(t)]=F_0[B_{F_0}(V(P_0),d)]$ is a connected subgraph of $F_0$.  By \cref{grow_in_forests}\cref{gif_zero}, $F_0[\Psi_0(t)]=G[\Psi_0(t)]$.

    \item For each $\ell\in\{1,\ldots,p\}$, $\Psi_\ell(t)$ is a ball, in $F_\ell$ around zero one or two connected subgraphs of $F_\ell$.  Thus $F_\ell[\Psi_\ell(t)]$ has at most two connected components.  By \cref{grow_in_forests}\cref{gif_ij}, $F_\ell[\Psi_{\ell}(t)]=G[\Psi_{\ell}(t)]$. \qedhere
  \end{enumerate}
  % Let $P_1eP_0e'P_2:=W_t$.
  %
  % \begin{enumerate}[(i)]
  %   \item
  %   \item  Since $t$ is admissible, $P_0\subseteq
  %   %  G-B_G(M'\cup \bigcup_{i=1}^pB_G(V(C_i),R-d),d)=
  %   F^--B_G(M',d)$.
  %   % \piotr{We dont remove $M'$ in the current definition. This is what I want to discuss.} \pat{Yes, but $P_0$ still avoids this ball because $t$ is admissible.}.
  %   Therefore, $B_G(V(P_0),d)\cap M'=\emptyset$ and, by the definition of $F^-$, $B_G(V(P_0),d)\cap B_G(V(C_i),R-d)=\emptyset$ for each $i\in\{1,\ldots,p\}$.  Therefore $B_G(V(P_0),d)\subseteq V(F)$.  Since $F$ is an induced subgraph of $G$, this implies that $B_G(V(P_0),d)=B_F(V(P_0),d)$ and that $G[B_G(V(P_0),d)]=F[B_F(V(P_0),d)]$. Since $P_0$ is a connected subgraph of $G$, $G[\Psi_0(t)]=G[B_G(V(P_0),d)]=F[B_F(V(P_0),d)]$ is connected.
  %
  %   \item If $\ell\not\in\{i,j\}$, $\Psi_\ell(t)=\emptyset$ so $G[\Psi_\ell(t)]$ is the empty graph, which has zero components.
  %
  %   If $\ell=i\neq j$ then, since $t$ is admissible $B_G(V(P_1),d))$ has no vertex in $Y_i\cup\{y_i\}$.  Therefore $G[\Psi_\ell(t)]=G[B_G(V(P_1),d)]= G[B_{G-Y_i\cup\{y_i\}}(V(P_1),d)]=(G-\{Y_i\cup\{y_i\})[B_{G-(Y_i\cup\{y_i\}}(V(P_1),d)=F_i[B_{F_i}(V(P_1),d)]$, where the last equality follows from the fact that $P_1$ contains a vertex of $C_i$.  Again,  $G[\Psi_\ell(t)]=G[B_G(V(P_1),d)]$ is connected because $P_1$ is a connected subgraph of $G$.  The case $\ell=j\neq i$ is handled by a symmetric argument that swaps the roles of $i$ and $j$ and considers $P_2$ rather than $P_1$.
  %
  %   Finally, in the case $\ell=i=j$, the same reasoning shows that $G[\Psi_\ell(t)]=F_i[B_{F_i}(V(P_1)\cup V(P_2),d]$, which has at most two components since each of $P_1$ and $P_2$ is a connnected subgraph of $F_i$. \qedhere
  % \end{enumerate}
\end{clmproof}

% \begin{clm}\label{s_path}
%   Let $t_1$ be an admissible tuple, let $P:=W_{t_1}$, let $t_2$ be an admissible tuple, let $Q:=W_{t_2}$ and let $S$ be any path of length at most $d$ with an endpoint $v\in V(P)$ and an endpoint in $w\in V(Q)$. Then $S\subseteq G-M'$
% \end{clm}
%
% \begin{clmproof}
%     Suppose, for the sake of contradiction, that $S$ contains a vertex $x\in M'$. Let $v$ be the endpoint of $S$ that lies in $P$ and $w$ be the endpoint of $S$ that lies in $Q$.  We claim that at least one of $t_1$ or $t_2$ is inadmissible.  This claim is immediate if at least one of $v$ or $w$ is in the first leg, second leg, or forest leg of its tuple.
%
%     Otherwise we may assume, without loss of generality that $v$ is in the first buffer of $t_1$ and $w$ is in the first buffer of $t_2$ and that $\dist_G(x,v)\le d/2$.  Let $P_1$ be the first leg of $t_1$ and $P_0$ be the forest leg of $t_1$.  Then $\dist_G(v,V(P_0))+\dist_G(v,V(P_1))\le d$.  Therefore $\dist_G(v,V(P_0))\le d/2$ or $\dist_G(v,V(P_1))\le d/2$.  In the former case $\dist_G(V(P_0),x)\le \dist_G(V(P_0),v)+\dist_G(v,x)\le d$, which would make $t_1$ inadmissible.  In the latter case $\dist_G(V(P_1),x)\le \dist_G(V(P_1),v)+\dist_G(v,x)\le d$, which would also make $t_1$ inadmissible.
% \end{clmproof}

\begin{clm}\label{overlap}
  Let $t:=(e,i,e',j)$ be an admissible good tuple and let $P_0$ be the forest leg of $W_t$.  Then $\Psi_0(t)\supseteq V(Q_t)\cap V(F)$.
\end{clm}

\begin{clmproof}
  Let $v$ be a vertex in $V(Q_t)\cap V(F)$.  If $v\in V(F^-)$ then $v\in V(P_0)\subseteq\Psi_0(t)$ and there is nothing to prove.  Now assume $v\in V(F)\setminus V(F^-)$. Therefore $v\in B_G(V(C_\ell),R-d)\setminus B_G(V(C_\ell),R-2d)$ for some $\ell\in\{1,\ldots,p\}$.  More specifically, $\ell=i$ if $v\in V(P_1)$ and $\ell=j$ if $v\in V(P_2)$. Without loss of generality, suppose $v\in V(P_1)$, so $\ell=i$. Then $P_0$ contains an endpoint $w$ of the $i$-exit edge $e$. Then $v\in B_G(w,d)\subseteq\Psi_0(t)$.
\end{clmproof}

\begin{clm}\label{w_distance}
  Let $t_1$ and $t_2$ be admissible good tuples with $\dist_G(V(W_{t_1}),V(W_{t_2}))\le d$.  Then $\Psi_i(t_1)\cap \Psi_i(t_2)\neq\emptyset$ for some $i\in\{0,\ldots,p\}$.
\end{clm}


% \pat{HMMMMM: Problems here caused by these stupid tuples with empty forest legs.  These still need to have a non-empty projection onto $F$.  Otherwise, the Hungarians have no way of knowing that the walk of $(e,i,d,j)$ is close to the walk of  $(e',i',d',j')$ when $|\{i,i',j,j'\}|=4$}

\begin{clmproof}
  Let $v\in V(W_{t_1})$ and $w\in V(W_{t_2})$ be such that $\dist_G(v,w)=\dist_G(V(W_{t_1}),V(W_{t_2}))$.
   % and let $S$ be a shortest path, in $G$, from $v$ to $w$.  Let $(e,i,e',j):=t$.

  We first consider the case in which at least one of $v$ is in the forest leg of $W_{t_1}$ or $w$ is in the forest leg of $W_{t_2}$.  Without loss of generality, suppose $v$ is in the forest leg $P_0$ of $W_{t_1}$. Then $w\in B_G(v,d)\subseteq B_G(V(P_0),d)\subseteq\Psi_0(t_1)\subseteq V(F)$.  By \cref{overlap}, $w\in \Psi_0(t_2)$.   Therefore $\Psi_0(t_1)\cap\Psi_0(t_2)\supseteq\{w\}\supsetneq\emptyset$.

  Next we consider the case where $v$ is not in the forest leg of $W_{t_1}$ and $w$ is not in the forest leg of $W_{t_2}$.  Without loss of generality, suppose $v$ is in the first leg $P_1$ of $W_{t_1}$ and $w$ is in the first leg $Q_1$ of $W_{t_2}$.  Let $(e,i,e',j):=t_1$, so $P_1\subseteq F_i^-$. Then $w\in B_G(v,d)\subseteq B_G(V(P_1),d)$ and, by \cref{grow_in_forests}\cref{gif_ij}, $B_G(V(P_1),d)=\Psi_i(t_1)\subseteq V(F_i)$.  By \cref{grow_in_forests}\cref{gif_unique_ends}, $Q_1\subseteq F_i$.  Therefore $\Psi_i(t_2)=B_{F_i}(V(Q_1),d)$.  Therefore $\Psi_i(t_1)\cap\Psi_i(t_2)\supseteq\{w\}\supsetneq \emptyset$. \qedhere
      %
      % If $w\notin B_G(V(C_i),R-2d)$ then $w\in V(F)$, so $w$ is in the first or second leg of $W_{t_2}$.  Assume, without loss of generality, that $w$ is in the first leg $Q_1$ of $W_{t_2}$.  Since $t_2$ is admissible and $w\in B_G(V(C_i),R)$, $w\not\in B_G(V(C_j),R)$ for each $j\in\{1,\ldots,p\}\setminus\{i\}$. Therefore $Q_1\subseteq B_G(V(C_i),R)$ and $\Psi_i(t_2)=B_G(V(Q_1),d)$.  Therefore $\Psi_i(t_1)\cap\Psi_i(t_2)\supseteq\{w\}\supsetneq \emptyset$.
      %
      %
      %
      % Therefore $w\in B_G(V(Q_0),d)$, where $Q_0$ is the forest leg of $t_2$.  Therefore $\Psi_0(t_1)\cap\Psi_0(t_2)\supseteq\{w\}\neq\emptyset$.
      %
      % \item $v$ lies in the first leg of $t_1$: Then let $C_i$ be the cycle on which the first leg of $t_1$ begins.  Then $w\in B_G(v,d)\subseteq\Psi_i(t_1)\subseteq B_G(V(C_i),R)$.  Therefore, $w$ is not a vertex of $T^-$.  Therefore $w\in B_G(V(C_j),R)$ where $C_j$ the cycle on which the first leg of $t_2$ begins or on which the second leg of $t_2$ ends. If $i\neq j$, then $w\in Y_{i,j}\subseteq M'$ which is not possible.  Therefore $i=j$, so $\Psi_i(t_1)\cap\Psi_i(t_2)\supseteq \{w\}\neq\emptyset$.
      %
      % % \item $v$ lies in the first buffer of $t_1$ and $w$ lies in the forest leg of $t_2$ then $S$ is contained in $F$.  Since $F$ is a forest, $S$ is the unique path from $v$ to $w$ in $F$.  Since $w\in V(F^-)$ and $v\not\in V(F^-)$, $S$ contains the edge $e$ in $P$ that joins the first and forest legs of $t_1$.  Therefore $\emptyset\neq e\subseteq \Phi_0(t_1)\cap\Phi_0(t_2)$.
      %
      % \item If $v$ is in the first buffer of $t_1$ and $w$ is in the first buffer of $t_2$ then let $C_i$ be the cycle on which the first leg of $t_1$ begins and let $C_j$ be the cycle on which the first leg of $t_2$ begins.
      %
      % If $i=j$ then there are two possibilities:  If $S$ contains a vertex $z$ in $F^-$ then $\Psi_0(t_1)\cap\Psi_0(t_2)\supseteq\{z\}\neq\emptyset$ and there is nothing more to prove.\pat{CAREFUL: Now we have to deal with the case in which $\Psi_{0}(t_1)$ or $\Psi_0(t_2)$ is the null graph!} Now assume that $S$ contains no vertex of $F^-$. Then, by \cref{s_path}, $S$ is a path in the unicyclic graph $G[B_{G-Y_i}(V(C_i),R)]$.   If $S$ contains a vertex $z$ in the first leg of $t_1$ then $w\in B_G(z,d)\subseteq\Psi_i(t_1)$, so $\Psi_i(t_1)\cap\Psi_i(t_2)\supseteq\{w\}\neq\emptyset$.  The same argument applies if $S$ contains a vertex $z$ in the first leg of $t_2$.  Otherwise, let $z$ be the vertex in $B_{G-Y_i}(V(C_i),R-d)$ that minimizes $\dist_{G-Y_i}(z,V(S))$.  Then $z$ is in the first leg of $t_1$ and in the first leg of $t_2$, so $z\in\Psi_i(t_1)\cap\Psi_i(t_2)$.
      %
      % If $i\neq j$ then there are two possibilities: If $S$ contains a vertex $z$ in $F^-$ then $z\in\Psi_0(t_1)\cap\Psi_0(t_2)$. Otherwise, $\dist_G(v,w)=\dist_{G-Y_i-Y_j}(v,w)\ge a+b$ where $a := R-\dist_G(V(C_i),v)$ and $b:=R-\dist_G(V(C_j),w)$. Let $P_0$ be the forest leg of $t_1$ and let $Q_0$ be the forest leg of $t_2$.  Then $\dist_G(V(P_0),v)\le a$ and $\dist_G(V(Q_0),w)\le b$.  Therefore $\dist_G(V(P_0),V(Q_0))\le 2a+2b\le 2d$.  Therefore, $\Psi_0(t_1)\cap\Psi_0(t_1)\neq\emptyset$.
  % \end{compactitem}
\end{clmproof}


\begin{clm}
  Let $X\subseteq V(G)$ and let $t:=(e,i,e',j)$ be an admissible tuple with extended forest leg $eP_0e'$ and such that $X\cap \Psi_\ell(t)\neq\emptyset$ for some $\ell\in\{0,\ldots,p\}$.  Then $B_G(X,R+d)\cap V(P_0)\neq\emptyset$.
\end{clm}

\begin{clmproof}
  If $\ell=0$ then, by definition $X$ contains a vertex in $B_G(V(P_0),d)=\Psi_0(t)$.  Therefore $B_G(X,d)\subseteq B_G(X,R+d)$ contains a vertex of $P_0$ and there is nothing more to prove.
  Otherwise, let $P_1$ be the first leg of $t$ and let $P_2$ be the second leg of $t$.
  If $\ell=i$ then $\dist_G(X,V(P_1))\le d$.  Therefore $B_G(X,d)$ contains a vertex of $P_1$.  Therefore $B_G(X,R+d)$ contains all vertices of $P_1$, including the endpoint of $e$ in $P_1$.  If $\ell=j$, then the same argument shows that $B_G(X,R+d)$ contains the endpoint of $e'$ in $P_2$.
\end{clmproof}

Now we can complete the proof. Consider the collection $S:=\{\Psi(t):\text{$t$ is an admissible tuple}\}$.  By \cref{thm:gyarfas-lehel}, either (a) there exists $\hat{S}\subseteq S$ of size at least $4k\log k+k$ such that for each distinct  $t_1,t_2\in S'$, $\Psi_\ell(t_1)\cap\Psi_\ell(t_2)=\emptyset$ for each $\ell\in\{0,\ldots,p\}$, or (b) there exists $\tilde{X}\subseteq V(G)$ of size at most $f(k)$ such that $\tilde{X}\cap(\bigcup_{\ell=0}^p \Psi_\ell(t))\neq\emptyset$ for each $t\in S$.

In case (b), we take the set
\[
   X:=\tilde{X}\cup X_M\cup \bigcup_{i=1}^p \left(X_i\cup \bigcup_{j=i+1}^p X_{i,j} \right)
\]
Then $|X|\le f(k) + O(k\log k)$.
By \cref{hit_cycle}, every cycle $C$ in $G$ contains a vertex in $B_G(X,2R+d)$.

In case (a), we let $S'$ denote the tuples $t$ in $\hat{S}$  such that $W_t$ contains a cycle.  If $|S'|\ge k$ then, by \cref{w_distance}, these cycles give a $d$-packing of $k$ cycles in $G$.  Let $S'':=\hat{S}\setminus S'$.  We apply Siminovits to the subgraph $G'$ of $G$ obtained by taking the union of $C_1,\ldots,C_p$ and $W_t$ for each $t\in S''$.  This gives a collection of $k$ vertex disjoint cycles $C_1',\ldots,C_k'$ in $G'$.  The same argument used to prove \cref{grow_unicycle}, along with \cref{w_distance}, shows that $C_1',\ldots,C_k'$ is a $d$-packing of cycles in $G$. BAM!
\end{proof}

\bibliographystyle{plainurlnat}
\bibliography{cep}
\end{document}
